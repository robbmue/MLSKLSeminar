\chapter{Einleitung}
\label{chap:einleitung}
Diese Seminararbeit beschäftigt sich mit häufigen Problemstellungen des Machine Learnings. Des weiteren wird sich mit den verschiedenen Methoden beschäftigt, welche Scikit zur Lösung dieser Probleme bereitstellt. Ziel dieser Arbeit ist es dem Leser ein besseres Verständnis für Machine Learning zu vermitteln. Es soll ihm auch näher bringen, welche Methoden von Scikit für verschiedene Problemstellungen am besten sind. Im ersten Teil der Seminararbeit wird zunächst ein generelles Verständnis für Machine Learning(ML) geschaffen. Im zweiten Teil werden verschiedene Scikit Funktionalitäten anhand des Beispiels “Boston housing Dataset” angewandt. Im letzten Teil der Arbeit sollen die verschiedenen Algorithmen miteinander verglichen werden.

\chapter{Motivation}
\label{sec:motivation}
Machine-Learning wird in der heutigen Zeit immer wichtiger und entwickelt sich immer weiter.\cite{MB} Machine-Learning findet sich schon heute in vielen Bereichen unseres Lebens wieder, wie z.B. in der Automobilindustrie. Vor allem im Komplex des Autonomen Fahrens steht die Industrie vor zahlreichen Herausforderungen, die es noch zu meistern gilt und von denen einige nur mit Hilfe von ML gelöst werden können. Eine einfache Aufgabe ist dabei noch das Erkennen von Straßenschildern. Eine Software zu schreiben, die Schilder unter allen möglichen Wetterbedingungen, Winkeln, Lichtverhältnissen etc. zuverlässig erkennt stellt eine unfassbar große Herausforderung dar, die sich jedoch mit ML relativ einfach meistern lässt, indem ein System auf entsprechend großen Datasets trainiert wird. 
Einen einfachen Einstieg in das Thema ML bieten Projekte wie SciKit-Learn. SciKit-Learn macht es jedem, der grundlegende Programmierkenntnisse besitzt, möglich selbst mit ML zu experimentieren, ohne die Algorithmen selbst implementieren zu müssen.



\section{Zielsetzung}
\label{sec:zielsetzung}
Ziel der Arbeit ist es, ein grundlegendes Verständnis für Machine Learning Projekte und Problematiken zu entwickeln und zu vermitteln.

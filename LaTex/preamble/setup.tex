% **************************** HYPERREF SETUP *******************************
\definecolor{linkcolor}{rgb}{1,0.5,0}
\hypersetup
{
bookmarks=true,                        % Lesezeichen im PDF erzeugen
bookmarksopen=true,                    % Lesezeichen im PDF sofort anzeigen
backref=true,                          % Rückverweise im Literaturverzeichnis
colorlinks=true,                       % Farbige Verweise
%hidelinks = true,                      % Verweise verbergen (entfernt Farbe und Rahmen)
pdfstartview={FitH},                   % Ansicht des PDFs beim öffnen
pdftitle={\titel},                     % Title des PDFs
pdfauthor={\author , \supervisor},     % Autor des PDFs
pdfsubject={\subject},                 % Thema des PDFs
%pdfcreator={Creator},                 % Erzeuger des Dokuments (Anwendungsprogramm)
%pdfproducer={Producer},               % Ersteller des PDFs (Programm/Bibliothek/Skript)
pdfkeywords={\keywords},               % Stichwörter zum PDF
linkcolor=linkcolor,                   % Farbe von Querverweisen
citecolor=green,                       % Farbe von Zitaten
filecolor=magenta,                     % Farbe von Verweisen auf Dateien
urlcolor=cyan                          % Farbe von URLs
}
% Weitere Optionen: http://www.tug.org/applications/hyperref/manual.html

% **************************** LISTINGS SETUP *******************************
\definecolor{keywords}{rgb}{0.5 0 0.3}
\definecolor{comments}{rgb}{0.25,0.5,0.37}
\lstset{ %
  backgroundcolor=\color{white},   % Hintergrundfarbe
  basicstyle=\linespread{0.94}\footnotesize\ttfamily, % Schrifteinstellungen für Quellcode
  breakatwhitespace=false,         % Automatische Zeilenumbrüche nur bei Leer- oder Tabulatorzeichen (Leerraum/whitespaces)
  breaklines=true,                 % Automatische Zeilenumbrüche
  captionpos=b,                    % Beschriftung unten
  commentstyle=\color{comments},   % Schrifteinstellungen für Kommentare
%  deletekeywords={...},            % Bestimmte Schlüsselwörter entfernen
  escapeinside={\%*}{*)},          % Defintion von Escape-Sequenzen
  extendedchars=true,              % Nicht ASCII-Zeichen erlauben
  frame=single,                    % Rahmen um den Quellcode
  keepspaces=true,                 % Einrückungen im Quellcode behalten
  keywordstyle=\bfseries\color{keywords},% Schrifteinstellungen für Schlüsselwörter
  language=java,                   % Programmiersprache des Quellcodes
%  morekeywords={*,...},            % Zusätzliche Schlüsselwörter
  numbers=left,                    % Zeilennummerierung
  numbersep=5pt,                   % Abstand zwischen Zeilennummerierung und Quellcode
  numberstyle=\tiny\color{gray}, % Schrifteinstellungen für Zeilennummern
  rulecolor=\color{black},         % if not set, the frame-color may be changed on line-breaks within not-black text (e.g. comments (green here))
  showspaces=false,                % Leerraum-Zeichen anzeigen
  showstringspaces=false,          % Leerzeichen in Zeichenketten anzeigen
  showtabs=false,                  % Tabulatorzeichen in Zeichenketten anzeigen
  stepnumber=1,                    % Schrittweite bei Zeilennummern
  stringstyle=\color{blue},        % Schrifteinstellungen für Zeichenketten
  tabsize=4,                       % Tabulatorbreite (Anzahl Leerzeichen)
  numberbychapter=false            % Nummeriere Quellcode fortlaufend je Kapitel
}
\renewcommand{\lstlistlistingname}{Quellcodeverzeichnis}
\renewcommand{\lstlistingname}{Quellcode}

\AtBeginDocument{\numberwithin{lstlisting}{section}} % Nummeriere Quellcode fortlaufend je Abschnitt

% ************************** HEADER/FOOTER SETUP ****************************
\pagestyle{scrheadings}

\clearscrheadings				% löscht voreingestellte Stile
\clearscrplain					% löscht voreingestellte Stile

\lohead[\headmark]{\headmark}
\rohead[\pagemark]{\pagemark}

\automark{chapter}

% **************************** GRAPHICX SETUP *********************************
\DeclareGraphicsExtensions{.pdf,.png,.jpg} % bekannte Graphik-Dateiformate (müssen nicht mehr im Dateinamen angegeben werden, also statt "beispiel.png" nur noch "beispiel")
\graphicspath{{./figure/}}   % path to graphics folder, usage {PATH},{ANOTHERPATH}...

% ************************** BIBLIOGRAPHY SETUP ********************************
\bibliographystyle{alphadin}    % Literaturverzeichnis nach DIN
%\AtBeginDocument{\nocite{*}} % Diese Zeile vor der Abgabe der Arbeit entfernen!

% ****************************** MATH SETUP ************************************
\everymath{\displaystyle}    % Erzwinge \displaystyle für Mathematischen-Modus


% ************************* THEOREMS AND PROOF *********************************
\newtheoremstyle{thesis}     % Name des neuen Theorem-Stils
{3pt}                        % Abstand oberhalb des Theorems
{3pt}                        % Abstand unterhalb des Theorems
{\itshape}                   % Schrifteinstellungen innerhalb des Theorems
{}                           % Einrückung der Theorem-Überschrift
{\bfseries}                  % Schrifteinstellungen für die Überschrift des Theorems
{}                           % Satzzeichen zwischen Überschrift und Theorem-Rumpf
{\newline}                   % Abstand hinter der Überschrift
{}                           % Spezifikation der Überschrift
  
\theoremstyle{thesis}        % Verwende neuen Theorem-Stil

\newtheorem{theorem}{Satz}[section] % neue Theorem-Umgebung: theorem (Satz)
\providecommand*{\theoremautorefname}{Satz} % autoref-Name für theorem

\newtheorem{definition}{Definition}[section] % neue Theorem-Umgebung: definition (Definition)
\providecommand*{\definitionautorefname}{Definition} % autoref-Name für definition

\renewcommand{\qedsymbol}{$\blacksquare$} % Schwarzes Quardrat als Symbol für: q. e. d.
\renewenvironment{proof}[1][\proofname]{{\bfseries #1:}~}{\qed} % "Beweise:" in Fettdruck

% *************************** PSEUDOCODE SETUP ********************************
\floatstyle{boxed}                        % Rahmen für pseudocode-Umgebung
\newfloat{pseudocode}{htbp}{lop}[section] % Definieren pseudocode-Umgebung
\floatname{pseudocode}{Pseudocode}        % Beschrifte pseudocode-Umgebung mit "Pseudocode"

\newcommand{\listofpseudocodename}{Pseudocodeverzeichnis}
\newcommand{\listofpseudocode}{\listof{pseudocode}{\listofpseudocodename}}
\providecommand*{\pseudocodeautorefname}{Pseudocode}

% ******************************* PAGE SETUP **********************************
\textheight23cm
\textwidth14cm
\voffset0cm
\topskip0cm
\topmargin-1.2cm
\headheight1.0cm
\headsep1.5cm
\oddsidemargin1.0cm
\evensidemargin1.0cm
\renewcommand{\baselinestretch}{1.4} 

% Hurenkinder und Schusterjungen verhindern
\clubpenalty = 10000
\widowpenalty = 10000
\displaywidowpenalty = 10000

% Verbesserung der Textsetzung
\tolerance = 1000
\emergencystretch = 20pt
%\chapter*{Überblick}
\section*{Kurzfassung}
In dieser Seminararbeit soll ein Einblick in die Welt des Machine Learning gegeben werden, wobei ein Oberthema Mobilität darstellt. 
Um das Thema am besten zu durchdringen wird zunächst ein Überblick über die vielfältigen Themenbereichen, Problemstellungen und Herangehensweisen geliefert. Außerdem werden zu jeder der drei Problematiken (Classification, Regression und Clustering) oberflächlich einige Algorithmen erklärt, um diese in der späteren Anwendung besser zu verstehen.
Als letzter Punkt des ersten Teils dieser Arbeit wird SciKit-Learn vorgestellt: eine Python-Bibliothek die einen simplen Einstieg in Machine Learning verspricht.
Mithilfe der von SciKit bereitgestellten Tools wird im zweiten Teil der Arbeit zunächst ein Experiment aus dem Buch “Hands-on Machine Learning with Scikit-Learn, Keras, and TensorFlow: Concepts, Tools, and Techniques to Build Intelligent Systems” (Aurélien Géron, 2019) nachgestellt, bei dem die Wohnungspreise in Kalifornien mit Machine Learning vorhergesagt werden sollen. Anschließend wird das Experiment um einige Aspekte erweitert, um vergleichen zu können, welche Algorithmen am besten sind. 
Nach, aber auch schon während, der Projektdurchführung hat sich herausgestellt, dass ein falscher Ansatz verfolgt wurde. Es lässt sich kein bester oder schlechtester Algorithmus definieren, vielmehr muss genau überlegt werden, welche Herangehensweise für die vorliegende Aufgabe die richtige ist, um das beste Ergebnis zu erhalten.

\vfill\vfill\vfill\vfill\vfill\vfill
\section*{Abstract}
This project paper offers a insight into the world of machine learning with some relatedness to the field of mobility. 
To give the best understanding about machine learning, initially there will be a brief overview about the diverse topics, different approaches, strategies and complexities. Besides you can find some shallow explanation of algorithms for the three basic problematics (classification, regression, clustering) to have a better understanding for them, while using them in the second part of our paper.
The last paragraph of the first part of this paper is an introduction to scikit-learn, which is a python-library that offers functions and methods to help with machine learning projects.
The second part of this paper is a whole machine learning project with the purpose to compare different kinds of algorithms, using scikit’s tools. Therefor an experiment from the book “Hands-on Machine Learning with Scikit-Learn, Keras, and TensorFlow: Concepts, Tools, and Techniques to Build Intelligent Systems” (Aurélien Géron, 2019) was postpositioned and extended, to make a comparison.
After, but even while, the execution of the project it became more and more clear that an incorrect approach has been used. Those algorithms cannot reasonably be compared to each other, because every different complexity demands a very different kind of strategy to master it. 
\vfill\vfill\vfill\vfill\vfill\vfill